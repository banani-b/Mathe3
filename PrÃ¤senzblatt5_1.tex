\documentclass[11pt,a4paper]{article}
%\usepackage[onehalfspacing]{setspace}
\usepackage [english]{babel}
\usepackage[utf8]{inputenc}
\usepackage{csquotes}                   
\usepackage{amsmath}
\usepackage{amssymb}
\usepackage{amsfonts}
\usepackage{amssymb}
\usepackage{faktor}
\usepackage[T1]{fontenc}
%\usepackage{pakete}
\usepackage{amsthm}
%\usepackage[most]{tcolorbox}
\newcounter{testexample}
\usepackage{xparse}
\usepackage{bbding} % Stift
\usepackage{varwidth}
\usepackage{color}
\usepackage{xcolor}
\usepackage{float}
\usepackage{tabularx}
\usepackage{booktabs}
\usepackage{bbm} % Gestrichene eins
\usepackage{bm} % Fette Symbole
\usepackage[font=footnotesize, justification=centering]{caption}
\usepackage{subcaption}
\usepackage[left=3.5cm,right=3.5cm,top=3.5cm,bottom=3.5cm]{geometry}
\usepackage{fancyhdr}

\pagestyle{fancy}
\fancyhf{}

%\rhead{ \includegraphics[scale=0.03]{unibonn.png}}
\lhead{\textsc{Präsenzblatt 4 Lösungen}}
\cfoot{\thepage}

\setlength{\textwidth}{14cm}

\usepackage{tikz}
	\usetikzlibrary{arrows,chains,matrix,positioning,scopes}
	\usetikzlibrary{intersections}
 	\usetikzlibrary{plotmarks}
    \usetikzlibrary{angles,quotes,babel}
    \usetikzlibrary{calc}
	\usetikzlibrary{shapes.misc}
	\usetikzlibrary{fadings,decorations.pathreplacing}
	\usetikzlibrary{cd}
%	\usepackage[arrow, matrix, curve]{xy}
\usepackage{pgfplots}
\pgfplotsset{compat=1.10}
\usepackage{subfiles}
\usepackage{stmaryrd}
\usepackage{enumerate}
\usepackage{graphicx}
\usepackage{enumitem}
\usepackage{mathtools}
\usepackage{tikz}
\usepackage{imakeidx}
\usepackage{abstract}
\usepackage{physics}
\usepackage[
	colorlinks=true,
	urlcolor=black,
	linkcolor=blue,
	citecolor=green
]{hyperref}
\newcommand{\ud}{\,\mathrm{d}}
\renewcommand{\familydefault}{\rmdefault}
\theoremstyle{definition} \newtheorem{theo}{Theorem}[section]
\theoremstyle{definition} \newtheorem{defi}{Definition}[section]
\theoremstyle{definition} \newtheorem{cor}{Corollary}[section]
\theoremstyle{definition} \newtheorem{lemmas}{Lemma}[section]
\theoremstyle{definition} \newtheorem{assp}{Assumption}[section]
\theoremstyle{definition} \newtheorem{exam}{Example}[section]	
\theoremstyle{definition} \newtheorem{sol}{Lösung}
\begin{document}


\paragraph{Aufgabe 1}
\begin{itemize} Sei $f:U\to \mathbb{C}$ holomorph für $U=U^{\ast}=\{ z\in \mathbb{C} | \overline{z}\in U\}$. 
	\item[a)] Sei $f(z)\in \mathbb{R}$ für $z\in U\cap \mathbb{R}$. Damit haben wir $z=\overline{z}\in U\cap \mathbb{R}$. Daraus sehen wir, dass $f(z)=f(\overline{z})\in \mathbb{R}$ für $z\in U\cap \mathbb{R}$ gilt. Also
	\begin{equation*}
	f(z)=\sum_{k=0}^{\infty}a_{k}(z-z_{0})^{k}=\overline{f(z)}=\sum_{k=0}^{\infty}\overline{a_{k}}(\overline{z}-\overline{z_{0}})^k=\sum_{k=0}^{\infty}\overline{a_{k}}(z-\overline{z_{0}})^k.
	\end{equation*}
	Dementsprechend muss $a_{k},z_{0}\in \mathbb{R}$ gelten. Wir sehen, dass $f(z)= \overline{f(\overline{z})}$ gilt für alle $z\in U$:
	\begin{equation*}
	f(z)=\sum_{k=0}^{\infty}a_{k}(z-z_{0})^{k}=\overline{\overline{\sum_{k=0}^{\infty}a_{k}(z-z_{0})^{k}}}=\overline{\sum_{k=0}^{\infty}a_{k}(\overline{z}-z_{0})^{k}}=\overline{f(\overline{z})}.
	\end{equation*}
	\item[b)] Folgt entweder aus i) und $-if(z)=\tilde{f}(z)$ oder genauer: für $z\in U\cap \mathbb{R}$ haben wir
	\begin{equation*}
	-\sum_{k=0}^{\infty}a_{k}(z-z_{0})^{k}=-f(z)=\overline{f(z)}=\overline{\sum_{k=0}^{\infty}a_{k}(z-z_{0})^{k}}= \sum_{k=0}^{\infty}\overline{a_{k}}(z-\overline{z_{0}})^{k},
	\end{equation*}
	also $z_{0}\in\mathbb{R}$ sowie $a_{k}\in i\mathbb{R}$. Für $z\in U$ folgt somit
	\begin{equation*}
	f(z)=\sum_{k=0}^{\infty}a_{k}(z-z_{0})^{k}=\overline{\overline{\sum_{k=0}^{\infty}a_{k}(z-z_{0})^{k}}}=-\overline \sum_{k=0}^{\infty}a_{k}(\overline{z}-z_{0})^{k}=-\overline{f(\overline{z})}.
	\end{equation*}
\end{itemize}

\paragraph{Aufgabe 2}
\begin{itemize}
	\item [a)] Entscheide, ob auf einem Gebiet, dass die Null enthält, eine holomorphe Funktion existiert, sodass 
	\[
	f\Big( \frac{1}{n} \Big) = (-1)^n \cdot \frac{1}{n}
	\]
	\textit{Lösung: } Solch eine Funktion existiert nicht. Um dies zu zeigen nutzen wir den Identitätssatz. Dieser soll zunächst in Erinnerung gerufen werden.
	\begin{description} \textsc{Identitätssatz:} Es seien $f$ und $g$ zwei auf einem Gebiet $U$ holomorphe Funktionen, $\exists z_0 \in U \textrm{ und } z_n  \rightarrow z_0: \forall n \in \mathbb{N} \quad z_n \ne z_0$ 
	und zusätzlich sei $f(z_n) = g(z_n) \quad \forall n \in \mathbb{N}$. 
	
	Dann gilt schon, dass $ f = g$ auf $U$.	
	\end{description}
	Sei nun $f$ wie oben beschrieben und $z_0 = 0$. 
	
	Wir finden zum einen eine holomorphe Funktion $g_1(z) = z$, die zusammen mit der Folge $z_n = \frac{1}{2n}$ die Bedingungen des Identitätssatzes erfüllt. Es gilt also $f(z) = z$.
	
	Gleichzeitig werden die Bedingungen auch mit einer Funktion $g_2(z) = -z$ für die Folge $z_n = \frac{1}{2n+1}$ erfüllt. Es gilt also auch $f(z) = -z$. Dies ist ein Widerspruch, folglich kann eine solche Funktion nicht existieren.
	
	\item[b )] Entscheide, ob auf einem Gebiet, dass die Null enthält eine holomorphe Funktion existiert, sodass 
	\[
	f\Big( \frac{1}{n} \Big) = \frac{1}{n^2 -1}.
	\]
	\textit{Lösung: } Solch eine Lösung existiert. Wir finden die Lösung
	\[
	f(z) = \frac{z^2}{1 -z^2}
	\]
	Diese Lösung ist holomorph in einer Umgebung der Null, denn $z^2$ und $1-z^2$ sind beide holomorph um $z = 0$ und $1 - z^2$ hat dort keine Nullstelle. Weiter gilt
	\[
	f \Big( \frac{1}{n} \Big) = \frac{\Big( \frac{1}{n} \Big)^2}{1 -\Big( \frac{1}{n} \Big)^2}
	 = \frac{1}{n^2 -1}
	\]
	Die Funktion $f$ erfüllt also alle geforderten Eigenschaften.
\end{itemize}


\paragraph{Aufgabe 3}

\begin{itemize}
	\item[a)] Bestimme die Ordnung der Nullstelle in $z=0$ folgender Funktionen: $$\tan(2z),\qquad \sin(z^{2})$$ 
	
	\textit{Lösung:} Für $\tan(2z)$ ergibt sich:
	\begin{align*}
		\tan(2z)\vert_{z=0}&=0\\
		\dv{z}\tan(2z)\vert_{z=0}&=\frac{2}{\cos^{2}(2z)}\vert_{z=0}=2\\
	\end{align*}
 	Damit hat $\tan(2z)$ bei $z=0$ eine Nullstelle 1. Ordnung.
	
	Für $\sin(z^{2})$ ergibt sich:
	\begin{align*}
		\sin(z^{2})\vert_{z=0}&=0\\
		\dv{z}\sin(z^{2})\vert_{z=0}&=2z\cos(z^{2})\vert_{z=0}=0\\
		\dv[2]{z}\sin(z^{2})&=(2\cos(z^{2})-4z^{2}\sin(z^{2}))\vert_{z=0}=2\\
	\end{align*}
	Also hat $\sin(z^{2})$ bei $z=0$ eine Nullstelle 2. Ordnung.
	
	\item[b)] Bestimme den Hauptteil der Laurentreihe in $a_{0,\pi}(0)$ von $\frac{1}{(\cos(z)-1)^{2}}$.
	
	\textit{Lösung:} Mit der Reihendarstellung des Cosinus gilt:
	\begin{align*}
		\frac{1}{(\cos(z)-1)^{2}}&=\frac{1}{(\sum_{n=0}^{\infty}(-1)^{n}\frac{z^{2n}}{(2n)!} - 1)^{2}}\\
		&=\frac{1}{(-\frac{z^{2}}{4}+\frac{z^{4}}{24}+\mathcal{O}(z^{6}))^{2}}\\
		&=\frac{4}{z^{4}}\cdot\frac{1}{(1-\frac{z^{2}}{12}+\mathcal{O}(z^{4}))^{2}}\\
		&=\frac{4}{z^{4}}\cdot \frac{1}{1-\frac{z^{2}}{6}+\mathcal{O}(z^{4})}\\
		&\overset{(\star)}{=}\frac{4}{z^{4}}\left(1+\frac{z^{2}}{6}+\mathcal{O}(z^{4})\right)\\
		&=\frac{4}{z^{4}}+\frac{2}{3z^{2}}+\mathcal{O}(1)
	\end{align*}
	Bei der Gleichung $(\star)$ wurde die geometrische Reihe für $\abs{\frac{z^{2}}{6}}<1$ (für $z\in (0,\pi)$) angewandt. 
\end{itemize}
\end{document}
