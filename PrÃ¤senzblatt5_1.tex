\documentclass[11pt,a4paper]{article}
%\usepackage[onehalfspacing]{setspace}
\usepackage [english]{babel}
\usepackage[utf8]{inputenc}
\usepackage{csquotes}                   
\usepackage{amsmath}
\usepackage{amssymb}
\usepackage{amsfonts}
\usepackage{amssymb}
\usepackage{faktor}
\usepackage[T1]{fontenc}
%\usepackage{pakete}
\usepackage{amsthm}
\usepackage[most]{tcolorbox}
\newcounter{testexample}
\usepackage{xparse}
\usepackage{bbding} % Stift
\usepackage{varwidth}
\usepackage{color}
\usepackage{xcolor}
\usepackage{float}
\usepackage{tabularx}
\usepackage{booktabs}
\usepackage{bbm} % Gestrichene eins
\usepackage{bm} % Fette Symbole
\usepackage[font=footnotesize, justification=centering]{caption}
\usepackage{subcaption}
\usepackage[left=3.5cm,right=3.5cm,top=3.5cm,bottom=3.5cm]{geometry}
\usepackage{fancyhdr}

\pagestyle{fancy}
\fancyhf{}

%\rhead{ \includegraphics[scale=0.03]{unibonn.png}}
\lhead{\textsc{Präsenzblatt 4 Lösungen}}
\cfoot{\thepage}

\setlength{\textwidth}{14cm}

\usepackage{tikz}
	\usetikzlibrary{arrows,chains,matrix,positioning,scopes}
	\usetikzlibrary{intersections}
 	\usetikzlibrary{plotmarks}
    \usetikzlibrary{angles,quotes,babel}
    \usetikzlibrary{calc}
	\usetikzlibrary{shapes.misc}
	\usetikzlibrary{fadings,decorations.pathreplacing}
	\usetikzlibrary{cd}
	\usepackage[arrow, matrix, curve]{xy}
\usepackage{pgfplots}
\pgfplotsset{compat=1.10}
\usepackage{subfiles}
\usepackage{stmaryrd}
\usepackage{enumerate}
\usepackage{graphicx}
\usepackage{enumitem}
\usepackage{mathtools}
\usepackage{tikz}
\usepackage{imakeidx}
\usepackage{abstract}
\usepackage{physics}
\usepackage[
	colorlinks=true,
	urlcolor=black,
	linkcolor=blue,
	citecolor=green
]{hyperref}
\newcommand{\ud}{\,\mathrm{d}}
\renewcommand{\familydefault}{\rmdefault}
\theoremstyle{definition} \newtheorem{theo}{Theorem}[section]
\theoremstyle{definition} \newtheorem{defi}{Definition}[section]
\theoremstyle{definition} \newtheorem{cor}{Corollary}[section]
\theoremstyle{definition} \newtheorem{lemmas}{Lemma}[section]
\theoremstyle{definition} \newtheorem{assp}{Assumption}[section]
\theoremstyle{definition} \newtheorem{exam}{Example}[section]	
		  \theoremstyle{definition} \newtheorem{sol}{Lösung}
\begin{document}



a) Sei $f(z)\in \mathbb{R}$ für $z\in U\cap \mathbb{R}$. Damit haben wir $z=\overline{z}\in U\cap \mathbb{R}$. Daraus sehen wir, dass $f(z)=f(\overline{z})\in \mathbb{R}$ für $z\in U\cap \mathbb{R}$ gilt. Also
\begin{equation*}
f(z)=\sum_{k=0}^{\infty}a_{k}(z-z_{0})^{k}=\overline{f(z)}=\sum_{k=0}^{\infty}\overline{a_{k}}(\overline{z}-\overline{z_{0}})^k=\sum_{k=0}^{\infty}\overline{a_{k}}(z-\overline{z_{0}})^k.
\end{equation*}
Dementsprechend muss $a_{k},z_{0}\in \mathbb{R}$ gelten. Wir sehen, dass $f(z)= \overline{f(\overline{z})}$ gilt für alle $z\in U$:
\begin{equation*}
f(z)=\sum_{k=0}^{\infty}a_{k}(z-z_{0})^{k}=\overline{\overline{\sum_{k=0}^{\infty}a_{k}(z-z_{0})^{k}}}=\overline{\sum_{k=0}^{\infty}a_{k}(\overline{z}-z_{0})^{k}}=\overline{f(\overline{z})}.
\end{equation*}
b) Folgt entweder aus i) und $-if(z)=\tilde{f}(z)$ oder genauer: für $z\in U\cap \mathbb{R}$ haben wir
\begin{equation*}
-\sum_{k=0}^{\infty}a_{k}(z-z_{0})^{k}=-f(z)=\overline{f(z)}=\overline{\sum_{k=0}^{\infty}a_{k}(z-z_{0})^{k}}= \sum_{k=0}^{\infty}\overline{a_{k}}(z-\overline{z_{0}})^{k},
\end{equation*}
also $z_{0}\in\mathbb{R}$ sowie $a_{k}\in i\mathbb{R}$. Für $z\in U$ folgt somit
\begin{equation*}
f(z)=\sum_{k=0}^{\infty}a_{k}(z-z_{0})^{k}=\overline{\overline{\sum_{k=0}^{\infty}a_{k}(z-z_{0})^{k}}}=-\overline \sum_{k=0}^{\infty}a_{k}(\overline{z}-z_{0})^{k}=-\overline{f(\overline{z})}.
\end{equation*}
\end{document}